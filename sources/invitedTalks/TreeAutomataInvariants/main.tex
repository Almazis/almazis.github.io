\documentclass{beamer}
%
% Choose how your presentation looks.
%
% For more themes, color themes and font themes, see:
% http://deic.uab.es/~iblanes/beamer_gallery/index_by_theme.html
%
\mode<presentation>
{
  \usetheme{Madrid}      % or try Darmstadt, Madrid, Warsaw, ...
  \usecolortheme{default} % or try albatross, beaver, crane, ...
  \usefonttheme{default}  % or try serif, structurebold, ...
  \setbeamertemplate{navigation symbols}{}
  \setbeamertemplate{caption}[numbered]
}

\usepackage[T2A]{fontenc}
\usepackage[utf8]{inputenc}
\usepackage[russian]{babel}
\usepackage{stmaryrd}
\usepackage{amsmath}
\usepackage{tabularx}
\usepackage{color, colortbl}
\usepackage{multirow}
\usepackage{tikz}
\usetikzlibrary{automata, arrows.meta, positioning, decorations.pathreplacing,calc,shapes,tikzmark}
\usepackage{subcaption}
\usepackage{caption}


\PassOptionsToPackage{table}{xcolor}
\definecolor{RowColorOdd}{rgb}{0.914,0.914,0.953}
\definecolor{RowColorEven}{rgb}{1,1,1}

\addtobeamertemplate{block example begin}{%
    \setlength{\textwidth}{0.4\textwidth}
}{}

\newcommand\myeq{\mathrel{\stackrel{\makebox[0pt]{\mbox{\normalfont\tiny def}}}{=}}}
\newcommand\mywedge{\mathrel{\stackrel{\makebox[0pt]{\mbox{\normalfont\tiny +}}}{\wedge}}}
\newcommand{\specialcell}[2][c]{%
  \begin{tabular}[#1]{@{}c@{}}#2\end{tabular}}
\newcommand{\cvc}{\textsc{CVC4}}
\newcommand{\eldarica}{\textsc{Eldarica}}
\newcommand{\spacer}{\textsc{Spacer}}
\newcommand{\zprover}{\textsc{Z3}}
\newcommand{\vampire}{\textsc{Vampire}}
\newcommand{\princessprover}{\textsc{Princess}}
\newcommand{\ringen}{\textsc{RInGen}}
\renewcommand{\phi}{\ensuremath{\varphi}}
\renewcommand{\emptyset}{\ensuremath{\varnothing}}
\newcommand{\tuple}[1]{\langle #1 \rangle}
\newcommand{\aut}{\mathcal{A}}
\newcommand{\redpause}{\addtocounter{beamerpauses}{-1}\pause\color<+>{red}}
\newcommand{\drive}
{
\begin{figure}[t]
\centering
\begin{tikzpicture}[scale=1.5]
% \draw[draw=black, rounded corners=20pt] (0,0) rectangle (5,3);
\draw (0,0) ellipse (16pt and 8pt);
\draw (0,0) node {a};
\draw (-1, 0) node {b};
\end{tikzpicture}
\end{figure}
}

\newcommand{\organization}
{
\begin{figure}[t]
\centering
\begin{tikzpicture}[scale=1.5]
\draw (0,0) node {Задачи СХД};
\draw (-3,0) -- (-1,0);
\draw (3,0) -- (1,0);
\draw (-3,0) [-stealth]-- (-3, -0.85);
\draw (3,0) [-stealth]-- (3, -0.85)
\draw (-3,-1) node {Доступность};
\draw (3,-1) node {Надежность};
\draw (0, -1) node {Производительность};
\draw (0, -0.2) [-stealth]-- (0, -0.85);
\end{tikzpicture}
\end{figure}
}

\newcommand{\dma}
{
\begin{figure}[t]
\centering
\begin{tikzpicture}[scale=.6]
\draw (0, 1) node {CPU};
\draw (0, -2) node {RAM};
\draw (0, -5) node {Block device};

\draw (-2.1, -4.3) rectangle (2.1, -5.7);
\draw (-2.1, -1.3) rectangle (2.1, -2.7);
\draw (-2.1, 1.7) rectangle (2.1, 0.3);


\draw (-1.3, -4.3) [-stealth]-- (-1.3,-2.7);
\draw (1.3, -4.3) [stealth-]-- (1.3,-2.7);

\draw (-1.3, -1.3) [-stealth]-- (-1.3, 0.3);
\draw (1.3, -1.3) [stealth-]-- (1.3, 0.3);
\end{tikzpicture}
\end{figure}
}

\newcommand{\nas}
{
\begin{figure}[t]
\centering
\begin{tikzpicture}[scale=.4]

\draw (-2.3, 2) rectangle (2.3, -6);
\draw (0, 1) node {CPU};
\draw (0, -2) node {RAM};
\draw (0, -5) node {Blk dev};
\draw (0, -6.7) node {Storage Node};

\draw (-2.1, -4.3) rectangle (2.1, -5.7);
\draw (-2.1, -1.3) rectangle (2.1, -2.7);
\draw (-2.1, 1.7) rectangle (2.1, 0.3);
\draw (-1.3, -4.3) [-stealth]-- (-1.3,-2.7);
\draw (1.3, -4.3) [stealth-]-- (1.3,-2.7);
\draw (-1.3, -1.3) [-stealth]-- (-1.3, 0.3);
\draw (1.3, -1.3) [stealth-]-- (1.3, 0.3);

\draw (2.3, -2) [stealth-]-- (4, -2);
\draw (2.3, -2) [-stealth]-- (4, -2);
\draw (6, -2) node {Network};
\draw (4, -1) rectangle (8 ,-3);

\draw (12, 0) node {User};
\draw (10.8, 0.6) rectangle (13.2, -0.6);
\draw (12, -4) node {User};
\draw (10.8, -3.4) rectangle (13.2, -4.6);

\draw (8, -1.8) [stealth-]-- (10.8, 0);
\draw (8, -1.8) [-stealth]-- (10.8, 0);

\draw (8, -2.2) [stealth-]-- (10.8, -4);
\draw (8, -2.2) [-stealth]-- (10.8, -4);

\end{tikzpicture}
\end{figure}
}

\newcommand{\bottleneck}
{
\begin{figure}[t]
\centering
\begin{tikzpicture}[scale=0.5]

\draw (0,0) [stealth-] -- (0, 13);

\draw (-0.5, 10) -- (0.5, 10);
\draw (-2.5, 10) node {\tiny 1999~г. SSE};

\node[align=left, anchor=west] at (1,11.5) 
    {CPU слабые и плохо справляются с логикой СХД\\ Логика СХД выносится в отдельный аппаратный контроллер \\(RAID-контроллер)};

\draw (-0.5, 5) -- (0.5, 5);
\draw (-2.5, 5) node {\tiny 2011~г. NVMe};

\node[align=left, anchor=west] at (1,7.5) 
    {На CPU появляются и развиваются векторные регистры \\ Бутылочным горлышком становится скорость операций с дисками \\ Логика на CPU позволяет внедрять дополнительную функциональность,\\ в том числе увеличивающую скорость операций};

\node[align=left, anchor=west] at (1,2.5) 
    {Появляются более быстрые диски и задачи в индустрии, для которых \\
    они необходимы (например запись видео высокого разрешения)\\
    Логика на CPU тормозит\\
    Ответ --- DPU (data processing unit)
    };



\end{tikzpicture}
\end{figure}
}

\newcommand{\stripe}
{
\begin{figure}[t]
\centering
\begin{tikzpicture}[scale=0.7]

\draw (0,0) rectangle (2,1);
\draw (2,0) rectangle (4,1);
\draw (5,0.5) node {$\ldots$};
\draw (6,0) rectangle (8,1);
\draw (8,0) rectangle (10,1);
\draw (10,0) rectangle (12,1);

\draw (1,0.5) node {$D_0$};
\draw (3,0.5) node {$D_1$};
\draw (7,0.5) node {$D_{N-1}$};
\draw (9,0.5) node {$P$};
\draw (11,0.5) node {$Q$};

\end{tikzpicture}
\captionsetup{labelformat=empty}
\caption{Страйп RAID 6}
\end{figure}
}

\tikzset{radiation/.style={{decorate,decoration={expanding waves,angle=90,segment length=4pt}}},
         disk/.pic={
        code={\tikzset{scale=7/10}
            \draw (1,1) ellipse (1cm and 0.5cm);
            \draw (0,0) arc (180:360:1cm and 0.5cm);%
            \draw (0,-1) arc (180:360:1cm and 0.5cm);%
            \draw (0,-2) arc (180:360:1cm and 0.5cm);%
            \draw (0,-3) arc (180:360:1cm and 0.5cm);%
            
            \draw (0,1) -- (0,-3.05);
            \draw (2,1) -- (2,-3.05);
            \draw (1,1) -- (1,2);
  }}
}

\newcommand{\raid}
{
\begin{figure}[t]
\centering
\begin{tikzpicture}

\path (0,0) pic {disk};
\path (2,0) pic {disk};
\draw (0.7,1.4) -- (2.7,1.4);
\draw (1.7 ,1.7) node {RAID 1};
\draw (0.7,0) node {$A_1$};
\draw (2.7,0) node {$A_1$};

\draw (0.7,-0.7) node {$B_1$};
\draw (2.7,-0.7) node {$B_1$};

\draw (0.7,-1.4) node {$C_1$};
\draw (2.7,-1.4) node {$C_1$};

\draw (0.7,-2.1) node {$\ldots$};
\draw (2.7,-2.1) node {$\ldots$};


\path (6,0) pic {disk};
\path (8,0) pic {disk};
\path (10,0) pic {disk};
\path (12,0) pic {disk};
\draw (6.7,1.4) -- (12.7,1.4);
\draw (9.7 ,1.7) node {RAID 5};


\draw (6.7,0) node {$A_1$};
\draw (8.7,0) node {$A_2$};
\draw (10.7,0) node {$A_3$};
\draw (12.7,0) node {$P_A$};

\draw (6.7,-0.7) node {$B_1$};
\draw (8.7,-0.7) node {$B_2$};
\draw (10.7,-0.7) node {$P_B$};
\draw (12.7,-0.7) node {$B_3$};

\draw (6.7,-1.4) node {$C_1$};
\draw (8.7,-1.4) node {$P_C$};
\draw (10.7,-1.4) node {$C_2$};
\draw (12.7,-1.4) node {$C_3$};

\draw (6.7,-2.1) node {$\ldots$};
\draw (8.7,-2.1) node {$\ldots$};
\draw (10.7,-2.1) node {$\ldots$};
\draw (12.7,-2.1) node {$\ldots$};

\end{tikzpicture}

\end{figure}
}


\newcommand{\raidPerf}
{
\begin{figure}[t]
\centering
\begin{tikzpicture}

\path (0,0) pic {disk};
\path (2,0) pic {disk};
\draw (0.7,1.4) -- (2.7,1.4);
\draw (1.7 ,1.7) node {RAID 1};
\draw (1.7 ,-3) node {Зеркалирование, может ускорять чтение};
\draw (0.7,0) node {$A_1$};
\draw (2.7,0) node {$A_1$};

\draw (0.7,-0.7) node {$B_1$};
\draw (2.7,-0.7) node {$B_1$};

\draw (0.7,-1.4) node {$C_1$};
\draw (2.7,-1.4) node {$C_1$};

\draw (0.7,-2.1) node {$\ldots$};
\draw (2.7,-2.1) node {$\ldots$};


\path (6,0) pic {disk};
\path (8,0) pic {disk};
\path (10,0) pic {disk};
\path (12,0) pic {disk};
\draw (6.7,1.4) -- (12.7,1.4);
\draw (9.7 ,1.7) node {RAID 0};
\draw (9.7 ,-3) node {Страйпинг, ускоряет чтение и запись};


\draw (6.7,0) node {$A_1$};
\draw (8.7,0) node {$A_2$};
\draw (10.7,0) node {$A_3$};
\draw (12.7,0) node {$A_4$};

\draw (6.7,-0.7) node {$B_1$};
\draw (8.7,-0.7) node {$B_2$};
\draw (10.7,-0.7) node {$B_3$};
\draw (12.7,-0.7) node {$B_4$};

\draw (6.7,-1.4) node {$C_1$};
\draw (8.7,-1.4) node {$C_2$};
\draw (10.7,-1.4) node {$C_3$};
\draw (12.7,-1.4) node {$C_4$};

\draw (6.7,-2.1) node {$\ldots$};
\draw (8.7,-2.1) node {$\ldots$};
\draw (10.7,-2.1) node {$\ldots$};
\draw (12.7,-2.1) node {$\ldots$};

\end{tikzpicture}

\end{figure}
}

\newcommand{\writehole}
{
\begin{figure}[t]
\centering
\begin{tikzpicture}[scale=0.5]

% \draw (0,0) rectangle (20,1);

\draw (0,0) rectangle (3.1,1);
\draw (1.55, -0.85) node {$D_1$};
\draw (1.55,1.2) [stealth-]-- (1.55, 2.5);
\draw[align=left, anchor=west] (1.6,2) node {Запись};

\draw[fill=green] (0.1,0.1) rectangle (0.3,0.9);
\draw[fill=green] (0.4,0.1) rectangle (0.6,0.9);
\draw[fill=green] (0.7,0.1) rectangle (0.9,0.9);
\draw[fill=green] (1,0.1) rectangle (1.2,0.9);
\draw(1.3,0.1) rectangle (1.5,0.9);
\draw (1.6,0.1) rectangle (1.8,0.9);
\draw (1.9,0.1) rectangle (2.1,0.9);
\draw (2.2,0.1) rectangle (2.4,0.9);
\draw (2.5,0.1) rectangle (2.7,0.9);
\draw (2.8,0.1) rectangle (3,0.9);

\draw (3.4, 0) rectangle (6.5,1);
\draw (4.95, -0.85) node {$D_2$};
\draw (4.95,1.2) [stealth-]-- (4.95, 2.5);
\draw[align=left, anchor=west] (5,2) node {Запись};


\draw[fill=green] (3.5,0.1) rectangle (3.7,0.9);
\draw[fill=green] (3.8,0.1) rectangle (4,0.9);
\draw[fill=green] (4.1,0.1) rectangle (4.3,0.9);
\draw[fill=green](4.4,0.1) rectangle (4.6,0.9);
\draw[fill=green] (4.7,0.1) rectangle (4.9,0.9);
\draw[fill=green] (5,0.1) rectangle (5.2,0.9);
\draw[fill=green] (5.3,0.1) rectangle (5.5,0.9);
\draw[fill=green] (5.6,0.1) rectangle (5.8,0.9);
\draw[fill=green] (5.9,0.1) rectangle (6.1,0.9);
\draw[fill=green] (6.2,0.1) rectangle (6.4,0.9);

% \draw (3.4, 0) rectangle (6.5,1);
\draw (6.8, 0) rectangle (9.9,1);
\draw (8.35, -0.85) node {$D_3$};


\draw (10.3,0) rectangle (13.4, 1);
\draw (11.85,1.2) [stealth-]-- (11.85, 2.5);
\draw[align=left, anchor=west] (11.9,2) node {Запись};
\draw (11.85,-0.85) node {$P$};
\draw[fill=green] (10.4,0.1) rectangle (10.6, 0.9);
\draw (10.7,0.1) rectangle (10.9, 0.9);
\draw (11,0.1) rectangle (11.2, 0.9);
\draw (11.3,0.1) rectangle (11.5, 0.9);
\draw (11.6,0.1) rectangle (11.8, 0.9);
\draw (11.9,0.1) rectangle (12.1, 0.9);
\draw (12.2,0.1) rectangle (12.4, 0.9);
\draw (12.5,0.1) rectangle (12.7, 0.9);
\draw (12.8,0.1) rectangle (13, 0.9);
\draw (13.1,0.1) rectangle (13.3, 0.9);


\end{tikzpicture}

\end{figure}
}

\newcommand{\san}
{
\begin{figure}[t]
\centering
\begin{tikzpicture}[scale=0.8]


% \draw (1, 0.5) node {Network}; 



\draw (3.25, 1.75) rectangle (5.25, 0.75);
\draw (4.25, 1.225) node {Node};

\draw (3.25, -0.75) rectangle (5.25, 0.25);
\draw (4.25, -0.225) node {Node};

% NAS
\draw (7, 0) rectangle (9, 1);
\draw (8, 0.5) node {Drives}; 
\draw[dashed] (5.8, -0.7) rectangle (9.2, 1.7);
\draw[anchor=south east] (9.1,-0.7) node {\tiny NAS};

% \draw (3,-5) rectangle (8,-2);



% \draw (3,-1) rectangle (8,2);

\draw (3.25, 3) rectangle (6.75, 4);
\draw (5, 3.5) node {Storage Node};

% \draw () elipse
\draw[dashed] (6.225,1.425) ellipse (130pt and 120pt);
\draw[fill=white] (1.5, -0.575) rectangle (2,3.425);
\node[rotate=90] (Nw) at (1.75, 1.425) {Network};

\node[draw] (u1) at (-1,0) {User};
\node[draw] at (-1,1.425) {User};
\node[draw] at (-1,2.85) {User};

%user- - nw
\draw (-0.4, 2.85) -- (1.5, 2.55);
\draw (-0.4, 1.425) -- (1.5, 1.425);
\draw (-0.4, 0) -- (1.5, 0.3);


\draw (5.25, -0.225) -- (7, 0.5);
\draw (5.25, 1.225) -- (7, 0.5);

\draw (3.25, -0.225) -- (2, 0.25);
\draw (3.25, 1.225) -- (2, 1.225);

\draw (3.25, 3.5) -- (2, 2.5);

\end{tikzpicture}

\end{figure}
}


\title[]{Представление инвариантов программ с алгебраическими типами данных в виде синхронных древовидных автоматов}
\author[Васенина А.~И.]{Васенина Анна Игоревна}
\institute[СПбГУ]{Санкт-Петербургский государственный университет \\ Кафедра системного программирования}
\date[16 ноября 2023 г.]{16 ноября 2023 г.}

\newenvironment{comment}{}{}

\begin{document}

\begin{frame}
  \titlepage
\end{frame}

% Uncomment these lines for an automatically generated outline.
%\begin{frame}{Outline}
%  \tableofcontents
%\end{frame}

\section{Introduction}

\begin{frame}{Условия верификации}
\begin{block}{Программа}
\textcolor{blue}{int} abs (\textcolor{blue}{int} x) \{

\quad \textcolor{blue}{if} x \geqslant 0

\quad \quad \textcolor{blue}{return} x 

\quad \textcolor{blue}{else}

\quad\quad \textcolor{blue}{return} -x

\}

\textcolor{blue}{assert}(abs(x) \geq 0) 
\end{block}
\pause
\begin{block}{Условия верификации}
\vspace{-1em}
\begin{align*}
    x  \geqslant 0 \wedge y = x &\to abs(x, y)\\
    x < 0 \wedge y = \text{-}x &\to abs(x, y)\\
    abs(x,y) \wedge y < 0 &\to \bot
\end{align*}
\end{block}

\end{frame}

\begin{frame}{Условия верификации}
\begin{exampleblock}{Функция}
$f: X \to Y$\\
$y = f(x)$
\end{exampleblock}
\begin{exampleblock}{График}
$G_f \subset X \times Y$ \\
$(x, y) \in G_f \iff y = f(x)$
\end{exampleblock}
\end{frame}

\begin{frame}{Условия верификации}
\begin{block}{Условия верификации}
В общем виде
\begin{center}
$\phi \wedge R_1(\overline{x}_1, y_1) \wedge R_2(\overline{x}_2, y_2) \wedge \ldots \wedge R_m(\overline{x}_m, y_m) \to R_0(\overline{x}_0, y_0)$
\end{center}
R_i \text{ --- неинтерпретированные предикатные символы, вызовы }\\
\quad\quad \text{ закодированных предикатными символами функций}\\
 
x_i \text{ --- входные значения}\\
y_i \text{ --- результаты}\\
\phi \text{ --- ограничение в логике первого порядка, оставшееся тело функции}
\end{block}
\begin{block}{Решение}
\begin{itemize}
    \item Контрпример
    \item Индуктивный инвариант
\end{itemize}
\end{block}
\end{frame}

\begin{frame}{Индуктивный инвариант}
\begin{block}{Программа с АТД}
\vspace{-1em}
\begin{align*}
    n = Z \wedge r = S(Z) &\to inc(n, r)\\
n = S(n') \wedge r = S(r') \wedge inc(n',r') &\to inc(n, r)\\
inc(n,r) \wedge r \neq S(n) &\to \bot
\end{align*}

\end{block}
\begin{block}{Теоретико-множественный инвариант}
$I = \{\langle Z, S(Z) \rangle, \langle S(Z), S(S(Z)) \rangle, \langle S(S(Z)), S(S(S(Z))) \rangle, \ldots \} $
\end{block}
\pause
Конечное представление
\pause
\begin{itemize}
    \item Формула логики первого порядка в теории АТД
    \item Формула логики первого порядка с ограничением на размер
    \item Язык автомата
    \item Язык синхронного автомата
\end{itemize}
\end{frame}

\begin{frame}{Формулы логики первого порядка}
\begin{block}{Определение}
$\phi ::=\ (t = t')\ |\ \neg \phi\ |\ \phi \wedge \phi\ |\ \phi \vee \phi\ |\ \exists x.\phi\ |\ \forall x.\phi$
\end{block}
\pause
\begin{block}{Теоретико-множественный инвариант}
$I = \{\langle Z, S(Z) \rangle, \langle S(Z), S(S(Z)) \rangle, \langle S(S(Z)), S(S(S(Z))) \rangle, \ldots \} $
\end{block}
\begin{block}{Конечное представление}
$inc(n, r) \iff r = S(n)$
\end{block}
\end{frame}

\begin{frame}{Формулы логики первого порядка}
\begin{alertblock}{Пример}
\vspace{-1.7em}
\begin{align*}
    x = Z &\to even(x)\\
x = S(S(y)) \wedge even(y) &\to even(x)\\
even(x) \wedge even(y) \wedge y = S(x) &\to \bot
\end{align*}

\end{alertblock}

Оценка выразительной силы
\begin{itemize}
    \item Процедура устранения кванторов\footnote{Oppen D. C. Reasoning about recursively defined data structures. ACM SIGACT-SIGPLAN Symposium on Principles of Programming Languages. – 1978}
    \pause
    \item После устранения кванторов: $x = S^n(Z),\ x = Z$
    \item Объединение конечных и коконечных множеств $\{x\ |\ x \neq 1 \wedge x \neq 2\} \cup \{x\ |\ x = 5 \vee x = 7\}$
    \pause
    \item Лемма о накачке\footnote{Kostyukov Y., Mordvinov D., Fedyukovich G. Beyond the elementary representations of program invariants over algebraic data types, ACM SIGPLAN International Conference on Programming Language Design and Implementation. – 2021}
\end{itemize}
\end{frame}

\begin{frame}{Ограничения на размер}
\begin{block}{Определение}
\begin{itemize}
    \item $size(t)$ --- количество конструкторов в $t$
    \item Пресбургеровские формулы над $size$
\end{itemize}
\end{block}
\pause
\begin{block}{Пример}
\vspace{-1em}
\begin{align*}
x = Z &\to even(x)\\
x = S(S(y)) \wedge even(y) &\to even(x)\\
even(x) \wedge even(y) \wedge y = S(x) &\to \bot
\end{align*}

\end{block}
\begin{block}{Конечное представление}
$even(x) \iff (size(x) + 1)\ |\ 2$
\end{block}

\end{frame}

\begin{frame}{Ограничения на размер}
\begin{alertblock}{Пример}
\vspace{-1em}
\begin{align*}
    x = Leaf &\to evenLeft(x)\\
x = Node(Node(x',y), z) \wedge evenLeft(x')  &\to evenLeft(x)\\
evenLeft(x) \wedge evenLeft(y) \wedge y =  Node(x, z) &\to \bot
\end{align*}

\end{alertblock}
Оценка выразительной силы
\begin{itemize}
    \item Лемма о накачке\footnote{Kostyukov Y., Mordvinov D., Fedyukovich G. Beyond the elementary representations of program invariants over algebraic data types, ACM SIGPLAN International Conference on Programming Language Design and Implementation. – 2021}
\end{itemize}
    
\end{frame}

\begin{frame}{Регулярные инварианты}
TODO: top-down и bottom-up автоматы + аналогия со строковыми
\end{frame}


\begin{frame}{Регулярные инварианты}
\begin{block}{Пример}
\vspace{-1em}
\begin{align*}
x = Z &\to even(x)\\
x = S(S(y)) \wedge even(y) &\to even(x)\\
even(x) \wedge even(y) \wedge y = S(x) &\to \bot    
\end{align*}

\end{block}
\begin{block}{Теоретико-множественный инвариант}
$I = \{Z,\ S(S(Z)),\ S(S(S(Z))),\ \ldots \} $
\end{block}
\begin{block}{Конечное представление}
\begin{minipage}{0.4\linewidth}
    $even(x) \iff x \in L$  
\end{minipage}
\begin{minipage}{0.4\linewidth}
\vspace{1em}
\evenAutomaton
\end{minipage}

\end{block}
\end{frame}



\begin{frame}{Регулярные инварианты}
\begin{block}{Общий случай АТД}
$T ::= C_0\ |\ C_1 (T)\ |\ C_2 (T, T)\ |\ \ldots\ |\ C_n(T,\ldots, T)$
\end{block}
\pause
\begin{block}{Древесное представление}
\begin{center}
   \treeView 
\end{center}

\end{block}
\end{frame}

\begin{frame}{Регулярные инварианты}
\begin{block}{Определение}
Автоматом над деревьями называют кортеж $\mathcal{A} = \langle Q, Q_F, \Delta \rangle$, где $Q$ --- конечное множество состояний, $Q_F \subseteq Q^n$ --- множество финальных состояний, $\Delta$ -- отношение перехода с правилами следующего вида:
\begin{align*}
    F(s_1, \ldots, s_m) \rightarrow s,
\end{align*}
\end{block}
\begin{block}{Определение}
Кортеж термов $\langle t_1, \ldots, t_n \rangle$ принимается автоматом $\mathcal{A}$ тогда и только тогда, когда, $\langle \mathcal{A}[t_1], \ldots, \mathcal{A}[t_n] \rangle \in Q_F$, где
\begin{align*}
    \mathcal{A} \big[F(t_{1},\ldots,t_{m}) \big] = \eqdef\left\{\begin{array}{rl}
    s, &\text{ если } \big(F (A[t_1],\ldots, A[t_m])\rightarrow s\big)\in\Delta,\\
    \bot, &\text{ иначе}.
    \end{array}\right.
\end{align*}
    
\end{block}
\end{frame}

\begin{frame}{Регулярные инварианты}
\begin{alertblock}{Пример}
\vspace{-1em}
\begin{align*}
    x = Z \wedge y = S(y') &\to lt(x, y)\\
    x = S(x') \wedge y = S(y') \wedge lt(x', y') &\to lt(x, y)\\
    lt(x, y) \wedge lt(y, x) &\to \bot
\end{align*}
\end{alertblock}
\begin{alertblock}{Пример}
\vspace{-1em}
\begin{align*}
    x = y \to Eq(x, y)
\end{align*}
\end{alertblock}
\pause
Оценка выразительной силы
\begin{itemize}
    \item Представление как декартового произведения унарных языков
    \pause
    \item Лемма о накачке\footnote{Comon H. et al. Tree automata techniques and applications. – 2008.}
\end{itemize}
\end{frame}

\begin{frame}{Выразительная сила представлений}
\vspace{1em}
\invariantReprComparison
\end{frame}

\begin{frame}{Синхронизированные древовидные автоматы}
\begin{block}{Определение}
    Синхронизированным древовидным автоматом называют автомат над новым алфавитом $(\Sigma_F \cup \{\bot\})^n$
\end{block}
\begin{block}{Стратегия синхронизации}
Стратегия синхронизации --- это инъективная функция из множества кортежей слов в исходном алфавите $\Sigma_F$ в слова в алфавите $(\Sigma_F \cup \{\bot\})^n$
\end{block}
\end{frame}

\begin{frame}{Синхронизированные древесные автоматы}
\begin{block}{Стандартная стратегия}
$f(s_1, \ldots, s_n) \oplus g(t_1, \ldots, t_m) = fg(s_1 \oplus t_1, \ldots, s_N \oplus t_N)$
\end{block}
\begin{block}{Пример}
\StandartSync
\end{block}
\end{frame}

\begin{frame}{Синхронизированные древесные автоматы}

\begin{block}{Пример}
\ListLen
%Length for cons - good, for snoc - bad\\
%В частности в тата доказано, что x = S(x') непредставим
\end{block}

\end{frame}



\begin{frame}{Синхронизированные древесные автоматы}
\begin{block}{Полная стратегия}
$f(s_1, \ldots, s_n) \otimes g(t_1, \ldots, t_m) = fg(s_1 \otimes t_1, \ldots, s_1 \otimes t_m, \ldots, s_n \otimes t_m)$
\end{block}
\begin{block}{Пример}
\FullSync
\end{block}

\end{frame}

\begin{frame}{Синхронизированные древесные автоматы}
Оценка выразительной силы
\begin{itemize}
\item Регулярные отношения
\pause
\item Отношение размера
\begin{itemize}
    \item Длина списка
\end{itemize}
\pause
\item Отношение равенства
\end{itemize}
\end{frame}


\begin{frame}{Синхронный автомат для равенства}
\begin{minipage}{1\linewidth}
$z = Node(x, y)$\\
\\
\PattternAutomata
\end{minipage}
\pause
\begin{minipage}{0.4\linewidth}
\vspace{1em}
\begin{align*}
    \aut'[\langle x, y, z \rangle]
\end{align*}
\end{minipage}
\begin{minipage}{0.4\linewidth}
\vspace{1em}
\begin{align*}
    \langle z,
    &\qquad \aut[\langle x, g \rangle] 
    \qquad \aut[\langle x, h \rangle] \\
    &\qquad \aut[\langle y, g \rangle] 
    \qquad \aut[\langle y, h \rangle] \rangle 
\end{align*}
\end{minipage}
\end{frame}




\begin{frame}{Синхронный автомат для равенства}
\begin{minipage}{1\linewidth}
$z = Node(x, y)$\\
\\
\PattternAutomata
\end{minipage}
\begin{minipage}{0.4\linewidth}
\vspace{1em}
\onslide\begin{align*}
    \aut'[\langle x, y, z \rangle]
\end{align*}
\end{minipage}
\begin{minipage}{0.4\linewidth}
\vspace{1em}
\begin{align*}
    &\langle z, \\
    &\qquad delta_{\aut}(x, g, x_1g_1, x_1g_2, x_2g_1, x_2g_2), \\
    &\qquad delta_{\aut}(x, h, x_1h_1, x_1h_2, x_2h_1, x_2h_2), \\
    &\qquad delta_{\aut}(y, g, y_1g_1, y_1g_2, y_2g_1, y_2g_2), \\
    &\qquad delta_{\aut}(y, h, y_1h_1, y_1h_2, y_2h_1, y_2h_2) \rangle.
\end{align*}
\end{minipage}
\end{frame}

\begin{frame}{Синхронный автомат для равенства}
\begin{minipage}{1\linewidth}
$z = Node(x, y)$\\
\\
\PattternAutomata
\end{minipage}
\begin{minipage}{0.4\linewidth}
\vspace{1em}
\begin{align*}
    delta_{\aut'} (x, y,& z, \\
    & x_1y_1z_1, x_1y_1z_2, \\
    & x_1y_2z_1, x_1y_2z_2, \\
    & x_2y_1z_1, x_2y_1z_2, \\
    & x_2y_2z_1, x_2y_2z_2).
\end{align*}
\end{minipage}
\begin{minipage}{0.4\linewidth}
\vspace{1em}
\begin{align*}
    &\langle z, \\
    &\qquad delta_{\aut}(x, g, x_1g_1, x_1g_2, x_2g_1, x_2g_2), \\
    &\qquad delta_{\aut}(x, h, x_1h_1, x_1h_2, x_2h_1, x_2h_2), \\
    &\qquad delta_{\aut}(y, g, y_1g_1, y_1g_2, y_2g_1, y_2g_2), \\
    &\qquad delta_{\aut}(y, h, y_1h_1, y_1h_2, y_2h_1, y_2h_2) \rangle.
\end{align*}
\end{minipage}

\pause
\begin{minipage}{1\linewidth}
\vspace{1em}
$x_1y_1z_1 \to x_1y_1g \to \langle g, x_1g_1, x_1g_2, y_1g_1, y_1g_2 \rangle$
\end{minipage}
\end{frame}

% \begin{minipage}{0.4\linewidth}
% \vspace{1em}

% \end{minipage}


\begin{frame}{Автоматический вывод}
\begin{block}{Условия верификации}
В общем виде
\begin{center}
$\phi \wedge R_1(\overline{x}_1, y_1) \wedge R_2(\overline{x}_2, y_2) \wedge \ldots \wedge R_m(\overline{x}_m, y_m) \to R_0(\overline{x}_0, y_0)$
\end{center}
R_i \text{ --- неинтерпретированные предикатные символы, вызовы }\\
\quad\quad \text{ закодированных предикатными символами функций}\\
 
x_i \text{ --- входные значения}\\
y_i \text{ --- результаты}\\
\phi \text{ --- ограничение в логике первого порядка, оставшееся тело функции}
\end{block}
\begin{block}{Ограничение}
$\phi ::=\ (t = t')\ |\ \neg \phi\ |\ \phi \wedge \phi\ |\ \phi \vee \phi\ |\ \exists x.\phi\ |\ \forall x.\phi$
\end{block}
\end{frame}

\begin{frame}{Языковая семантика}
\begin{block}{Определение}
\small Языковую семантику бескванторной формулы определим индуктивно
\begin{minipage}{0.48\linewidth}
    \begin{itemize}
        \item $L(p) \subseteq T(\Sigma)^k$
        \item $L\llbracket \phi_1 \wedge \phi_2 \rrbracket = L \llbracket \phi_1 \rrbracket \cap L \llbracket \phi_2 \rrbracket$
    \end{itemize}
\end{minipage}
\hfill
\begin{minipage}{0.48\linewidth}
    \begin{itemize}
        \item $L\llbracket \neg \phi \rrbracket = T(\Sigma)^n \setminus L \llbracket \phi \rrbracket$
        \item $L\llbracket \phi_1 \vee \phi_2 \rrbracket = L \llbracket \phi_1 \rrbracket \cup L \llbracket \phi_2 \rrbracket$
    \end{itemize}
\end{minipage}
\begin{itemize}
    \item $L\llbracket p(\overline{t}(x_1, \ldots, x_n)) \rrbracket = \{ u \in T(\Sigma)^n \ | \ t[u] \in L\llbracket p \rrbracket \}$
\end{itemize}
\end{block}


\begin{block}{Определение}
\small $L \models \phi \iff L \llbracket \neg \phi \rrbracket = \emptyset$ 
\end{block}

\begin{block}{Теорема}
\small Бескванторная формула $\phi$ выполнима в языковой семантике тогда и только тогда, когда она выполнима в семантике Тарского
\end{block}

\end{frame}

\begin{frame}{Синхронные древовидные автоматы}
\begin{block}{Определение}
\small Языковую семантику бескванторной формулы определим индуктивно
\begin{itemize}
        \item $L(p) \subseteq T(\Sigma)^k$
        \item $L\llbracket \phi_1 \wedge \phi_2 \rrbracket = L \llbracket \phi_1 \rrbracket \cap L \llbracket \phi_2 \rrbracket$
        \item $L\llbracket \neg \phi \rrbracket = T(\Sigma)^n \setminus L \llbracket \phi \rrbracket$
        \item $L\llbracket \phi_1 \vee \phi_2 \rrbracket = L \llbracket \phi_1 \rrbracket \cup L \llbracket \phi_2 \rrbracket$ 
        \item \textcolor<2-3>{red}{$L\llbracket p(\overline{t}(x_1, \ldots, x_n)) \rrbracket = \{ u \in T(\Sigma)^n \ | \ t[u] \in L\llbracket p \rrbracket \}$} 
\end{itemize}
\end{block}
\pause

\onslide<2>\begin{block}{Нижний остаток (downward quotient)}
\vspace{-1em}
\begin{center}
    \begin{align*}
    t_1(x_{11}, \ldots, x_{1N_1}), \ldots, t_K(x_{K1}, \ldots x_{KN_K}) \in L\llbracket p \rrbracket  \iff \\
    u_{11}, \ldots, u_{1N_1}, \ldots, u_{K1}, \ldots u_{KN_K} \in L\llbracket p(\overline{t}(x_{11}, \ldots, x_{KN_K})) \rrbracket
    \end{align*}
\end{center}

\end{block}

\pause
\vspace{-6.5em}
\begin{block}{Теорема}
Класс языков, представимых синхронными древовидными автоматами \textit{с полной сверткой} замкнут относительно операции взятия нижнего остатка
\end{block}
\end{frame}


\begin{frame}{Синхронные древовидные автоматы}
\begin{block}{Логическая программа}
\vspace{-1em}
\begin{align*}
x = Z \wedge y = S(y') &\to lt(x, y)\\
x = S(x') \wedge y = S(y') \wedge lt(x', y') &\to lt(x, y)\\
lt(x, y) \wedge lt(y, x) &\to \bot
\end{align*}
\end{block}

\begin{block}{Инвариант}
\begin{minipage}{0.4\linewidth}
\vspace{-1em}
    \begin{align*}
    \aut = \{Q&, Init, Q_F, \delta\}\\
        Q &= \{q_0, q_1, q_2\} \\
        Init &= q_0\\
        Q_F &= \{q_1\}\\
        \mathcal{L}(\aut) = \{\tuple{&Z, S(Z)}, \tuple{Z, S(S(Z))}, ...\}
    \end{align*}

\end{minipage}
\begin{minipage}{0.4\linewidth}
\vspace{1em}
\ltAutomaton
\end{minipage}
\end{block}
\end{frame}





\begin{frame}{Вывод инвариантов}
\workflow
    % \begin{figure}[htbp]
    %     \centering
    %         \includegraphics[width=300px]{pic/invariants.png}
    %     \label{fig:somthing}
    % \end{figure}

\end{frame}

\begin{frame}{Эксперименты}

% Добавить конфигурацию сервера?
SAT --- number of derived invariants

UNSAT --- number of derived counterexamples

$\exists!$ --- number of unique results

\begin{table}[h]
\resizebox{\textwidth}{!}{%
    \centering
    \begin{tabular}{ |c|c|r|c|c|c|c|  }
     \hline
     
     Data set & \# & Result & \spacer{} & \eldarica{} & \ringen{} & \textsc{RInGen-TTA}\\
     \hline
     
     \multirow{4}{*}{\emph{TIP}} & \multirow{4}{*}{454}
                                & SAT                   & 26    & 46    & 25    & 43     \\
                                && $\exists !$ SAT      & 7     & 14    & 0    & 4   \\
                                && UNSAT                & 22    & 12    & 21    & 21   \\
                                && $\exists !$ UNSAT    & 7     & 0     & 0     & 0    \\
     \hline
    \end{tabular}}
\end{table}
\centering
Результаты экспериментов на весну 2022 г.


\end{frame}

\begin{frame}{Коллаборативные инварианты}

Коллаборативным инвариантом\footnote{Kostyukov Y., Mordvinov D., Fedyukovich G. Collaborative Inference of Combined Invariants //Proceedings of 24th International Conference on Logic. – 2023. – Т. 94. – С. 288-305.} называется формула вида
\begin{align*}
    \phi(\overline{x}) \vee \overline{x} \in L,
\end{align*}
где $\phi$ --- формула первого порядка, а $L$ некоторый формальный язык

\begin{table}[h]
\resizebox{\textwidth}{!}{%
    \centering
    \begin{tabular}{ |c|c|r|c|c|c|  }
     \hline
     
     Data set & \# & Result & \spacer{} & \ringen{} & \textsc{Collaborative}\\
     \hline
     
     \multirow{2}{*}{\emph{TIP}} & \multirow{2}{*}{454}
                                & SAT                   & 20    & 135    &   189   \\
                                && UNSAT                & 15    & 46    &  28  \\
     \hline
    \end{tabular}}    
\end{table}
\centering
Результаты экспериментов на лето 2023 г.

\end{frame}

\end{document}
